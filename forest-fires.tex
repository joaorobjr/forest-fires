% Options for packages loaded elsewhere
\PassOptionsToPackage{unicode}{hyperref}
\PassOptionsToPackage{hyphens}{url}
\PassOptionsToPackage{dvipsnames,svgnames*,x11names*}{xcolor}
%
\documentclass[
  12pt,
]{report}
\usepackage{lmodern}
\usepackage{amssymb,amsmath}
\usepackage{ifxetex,ifluatex}
\ifnum 0\ifxetex 1\fi\ifluatex 1\fi=0 % if pdftex
  \usepackage[T1]{fontenc}
  \usepackage[utf8]{inputenc}
  \usepackage{textcomp} % provide euro and other symbols
\else % if luatex or xetex
  \usepackage{unicode-math}
  \defaultfontfeatures{Scale=MatchLowercase}
  \defaultfontfeatures[\rmfamily]{Ligatures=TeX,Scale=1}
  \setmainfont[]{Source Variable Pro}
  \setsansfont[]{Source Sans Pro}
\fi
% Use upquote if available, for straight quotes in verbatim environments
\IfFileExists{upquote.sty}{\usepackage{upquote}}{}
\IfFileExists{microtype.sty}{% use microtype if available
  \usepackage[]{microtype}
  \UseMicrotypeSet[protrusion]{basicmath} % disable protrusion for tt fonts
}{}
\makeatletter
\@ifundefined{KOMAClassName}{% if non-KOMA class
  \IfFileExists{parskip.sty}{%
    \usepackage{parskip}
  }{% else
    \setlength{\parindent}{0pt}
    \setlength{\parskip}{6pt plus 2pt minus 1pt}}
}{% if KOMA class
  \KOMAoptions{parskip=half}}
\makeatother
\usepackage{xcolor}
\IfFileExists{xurl.sty}{\usepackage{xurl}}{} % add URL line breaks if available
\IfFileExists{bookmark.sty}{\usepackage{bookmark}}{\usepackage{hyperref}}
\hypersetup{
  pdftitle={Forest Fires in Portugal - What Are The Causes?},
  pdfauthor={By Robson Teixeira, Eduardo Rodrigues and Claudio Rocha},
  colorlinks=true,
  linkcolor=blue,
  filecolor=Maroon,
  citecolor=Blue,
  urlcolor=blue,
  pdfcreator={LaTeX via pandoc}}
\urlstyle{same} % disable monospaced font for URLs
\usepackage[top=25mm,bottom=25mm,left=25mm,right=20mm,heightrounded]{geometry}
\usepackage{longtable,booktabs}
% Correct order of tables after \paragraph or \subparagraph
\usepackage{etoolbox}
\makeatletter
\patchcmd\longtable{\par}{\if@noskipsec\mbox{}\fi\par}{}{}
\makeatother
% Allow footnotes in longtable head/foot
\IfFileExists{footnotehyper.sty}{\usepackage{footnotehyper}}{\usepackage{footnote}}
\makesavenoteenv{longtable}
\usepackage{graphicx,grffile}
\makeatletter
\def\maxwidth{\ifdim\Gin@nat@width>\linewidth\linewidth\else\Gin@nat@width\fi}
\def\maxheight{\ifdim\Gin@nat@height>\textheight\textheight\else\Gin@nat@height\fi}
\makeatother
% Scale images if necessary, so that they will not overflow the page
% margins by default, and it is still possible to overwrite the defaults
% using explicit options in \includegraphics[width, height, ...]{}
\setkeys{Gin}{width=\maxwidth,height=\maxheight,keepaspectratio}
% Set default figure placement to htbp
\makeatletter
\def\fps@figure{htbp}
\makeatother
\setlength{\emergencystretch}{3em} % prevent overfull lines
\providecommand{\tightlist}{%
  \setlength{\itemsep}{0pt}\setlength{\parskip}{0pt}}
\setcounter{secnumdepth}{5}
\usepackage{caption} \usepackage{fancyhdr} \usepackage{lmodern} \usepackage[detect-all]{siunitx}

\title{Forest Fires in Portugal - What Are The Causes?}
\usepackage{etoolbox}
\makeatletter
\providecommand{\subtitle}[1]{% add subtitle to \maketitle
  \apptocmd{\@title}{\par {\large #1 \par}}{}{}
}
\makeatother
\subtitle{Practical Assignment of Data Mining I}
\author{By Robson Teixeira, Eduardo Rodrigues and Claudio Rocha}
\date{M:CC -- FCUP, 10/01/2021}

\begin{document}
\maketitle

{
\hypersetup{linkcolor=}
\setcounter{tocdepth}{1}
\tableofcontents
}
\begin{center}\rule{0.5\linewidth}{0.5pt}\end{center}

\hypertarget{abstract}{%
\chapter*{Abstract}\label{abstract}}
\addcontentsline{toc}{chapter}{Abstract}

\hypertarget{introduction}{%
\chapter{Introduction}\label{introduction}}

\hypertarget{problem-definition}{%
\chapter{Problem Definition}\label{problem-definition}}

Forest fires are a very important issue that negatively affects climate
change. Typically, the causes of forest fires are those oversights,
accidents and negligence committed by individuals, intentional acts and
natural causes. The latter is the root cause for only a minority of the
fires.

Their harmful impacts and effects on ecosystems can be major ones. Among
them, we can mention the disappearance of native species, the increase
in levels of carbon dioxide in the atmosphere, earth's nutrients
destroyed by the ashes, and the massive loss of wildlife.

Data mining techniques can help in the prediction of the cause of the
fire and, thus, better support the decision of taking preventive
measures in order to avoid tragedy. In effect, this can play a major
role in resource allocation, mitigation and recovery efforts.

\hypertarget{forest-fire-dataset}{%
\chapter{Forest Fire Dataset}\label{forest-fire-dataset}}

The ICFN - Nature and Forest Conservation Institute has the record of
the list of forest fires occurred in Portugal for several years. For
each fire, there is information such as the site, the alert date/hour,
the extinction date/hour, the affected area and the cause type. A
classifications for causes types are presented in table
@ref(tab:cause\_type).

Table 1: (\#tab:cause\_type) Classifications of causes of forest fires.

\begin{longtable}[]{@{}ll@{}}
\toprule
\begin{minipage}[b]{0.11\columnwidth}\raggedright
Cause\strut
\end{minipage} & \begin{minipage}[b]{0.83\columnwidth}\raggedright
Description\strut
\end{minipage}\tabularnewline
\midrule
\endhead
\begin{minipage}[t]{0.11\columnwidth}\raggedright
Unknown\strut
\end{minipage} & \begin{minipage}[t]{0.83\columnwidth}\raggedright
absence of suficient objective evidence to determine the cause of the
ignition of fire\strut
\end{minipage}\tabularnewline
\begin{minipage}[t]{0.11\columnwidth}\raggedright
Natural\strut
\end{minipage} & \begin{minipage}[t]{0.83\columnwidth}\raggedright
lightning generated in thunderstorms\strut
\end{minipage}\tabularnewline
\begin{minipage}[t]{0.11\columnwidth}\raggedright
Negligence\strut
\end{minipage} & \begin{minipage}[t]{0.83\columnwidth}\raggedright
the misguided use of fire in activities such as burning trash, mass
burning of agricultural and forest fuels, fun and leisure activities;
failure to properly extinguish cigarettes by smokers; the dispersal and
transport of incandescent particles from chimneys; etc.\strut
\end{minipage}\tabularnewline
\begin{minipage}[t]{0.11\columnwidth}\raggedright
Intentional\strut
\end{minipage} & \begin{minipage}[t]{0.83\columnwidth}\raggedright
incendiarism and arson, mostly resulting from behaviors and attitudes
reacting to theconstraints of agroforestry management systems and to
conflicts related to land use\strut
\end{minipage}\tabularnewline
\begin{minipage}[t]{0.11\columnwidth}\raggedright
Rekindling\strut
\end{minipage} & \begin{minipage}[t]{0.83\columnwidth}\raggedright
reburning of an area over which a fire has previously passed, but where
fuel has been left that is later ignited by latent heat, sparks, or
embers\strut
\end{minipage}\tabularnewline
\bottomrule
\end{longtable}

\#-----------------------------XXXXX----------------------------

\#------------------------------XXXX---------------------------------

The dataset used in this study was provided by ICFN, and it contains the
data on reported forest fires during 2015 and its respective causes. The
data are distributed in files:

\begin{itemize}
\tightlist
\item
  \textbf{fires2015train.csv} --- the file contain the data of 7511
  reported forest fires during 2015
\end{itemize}

A summary of the structure of it and a glimpse of their first rows is
provided below.

\begin{verbatim}
## Rows: 7,511
## Columns: 21
## $ id                 <int> 1, 2, 3, 4, 5, 6, 7, 8, 9, 10, 11, 12, 13, 14, 1...
## $ region             <chr> "Entre Douro e Minho", "Entre Douro e Minho", "T...
## $ district           <chr> "Viana do Castelo", "Porto", "Vila Real", "Vila ...
## $ municipality       <chr> "Ponte de Lima", "Marco de Canaveses", "Boticas"...
## $ parish             <chr> "Serdedelo", "Vila Boa de Quires", "Cerdedo", "G...
## $ lat                <chr> "41:44:48.5663999999878''", "41:12:58.4280000000...
## $ lon                <chr> "8:31:12.3276000000027''", "8:12:28.378800000002...
## $ origin             <chr> "fire", "fire", "fire", "firepit", "firepit", "f...
## $ alert_date         <chr> "2015-03-24", "2015-03-24", "2015-03-24", "2015-...
## $ alert_hour         <chr> "17:01:00", "17:10:00", "21:40:00", "16:00:00", ...
## $ extinction_date    <chr> "2015-03-24", "2015-03-24", "2015-03-25", "2015-...
## $ extinction_hour    <chr> "18:09:00", "18:47:00", "05:45:00", "17:00:00", ...
## $ firstInterv_date   <chr> "2015-03-24", "2015-03-24", "2015-03-24", "2015-...
## $ firstInterv_hour   <chr> "17:10:00", "17:16:00", "22:00:00", "16:14:00", ...
## $ alert_source       <lgl> NA, NA, NA, NA, NA, NA, NA, NA, NA, NA, NA, NA, ...
## $ village_area       <dbl> 2.50, 0.00, 0.50, 0.00, 0.10, 0.00, 0.35, 0.50, ...
## $ vegetation_area    <dbl> 0.000, 1.350, 38.000, 0.010, 0.000, 0.100, 14.82...
## $ farming_area       <dbl> 0.0000, 0.0000, 0.0000, 0.0000, 0.0000, 0.0000, ...
## $ village_veget_area <dbl> 2.500, 1.350, 38.500, 0.010, 0.100, 0.100, 15.17...
## $ total_area         <dbl> 2.5000, 1.3500, 38.5000, 0.0100, 0.1000, 0.1000,...
## $ cause_type         <chr> "negligent", "negligent", "negligent", "negligen...
\end{verbatim}

\begin{verbatim}
## # A tibble: 6 x 21
##      id region district municipality parish lat   lon   origin alert_date
##   <int> <chr>  <chr>    <chr>        <chr>  <chr> <chr> <chr>  <chr>     
## 1     1 Entre~ Viana d~ Ponte de Li~ Serde~ 41:4~ 8:31~ fire   2015-03-24
## 2     2 Entre~ Porto    Marco de Ca~ Vila ~ 41:1~ 8:12~ fire   2015-03-24
## 3     3 Trás-~ Vila Re~ Boticas      Cerde~ 41:3~ 07:5~ fire   2015-03-24
## 4     4 Trás-~ Vila Re~ Montalegre   Gralh~ 41:5~ 7:42~ firep~ 2015-03-25
## 5     5 Trás-~ Vila Re~ Valpaços     Alger~ 41:3~ 07:2~ firep~ 2015-03-12
## 6     6 Entre~ Vila Re~ Mondim de B~ Ermelo 41:2~ 07:5~ firep~ 2015-03-13
## # ... with 12 more variables: alert_hour <chr>, extinction_date <chr>,
## #   extinction_hour <chr>, firstInterv_date <chr>, firstInterv_hour <chr>,
## #   alert_source <lgl>, village_area <dbl>, vegetation_area <dbl>,
## #   farming_area <dbl>, village_veget_area <dbl>, total_area <dbl>,
## #   cause_type <chr>
\end{verbatim}

\begin{verbatim}
##              variable q_zeros p_zeros q_na   p_na q_inf p_inf      type unique
## 1                  id       0    0.00    0   0.00     0     0   integer   7511
## 2              region       0    0.00  501   6.67     0     0 character     10
## 3            district       0    0.00    0   0.00     0     0 character     19
## 4        municipality       0    0.00    0   0.00     0     0 character    297
## 5              parish       0    0.00    0   0.00     0     0 character   2270
## 6                 lat       0    0.00    0   0.00     0     0 character   5858
## 7                 lon       0    0.00    0   0.00     0     0 character   5867
## 8              origin       0    0.00    0   0.00     0     0 character      5
## 9          alert_date       0    0.00    0   0.00     0     0 character    317
## 10         alert_hour       0    0.00    0   0.00     0     0 character   1312
## 11    extinction_date       0    0.00    9   0.12     0     0 character    319
## 12    extinction_hour       0    0.00    9   0.12     0     0 character   1201
## 13   firstInterv_date       0    0.00  214   2.85     0     0 character    318
## 14   firstInterv_hour       0    0.00  215   2.86     0     0 character   1202
## 15       alert_source       0    0.00 7511 100.00     0     0   logical      0
## 16       village_area    5349   71.22    0   0.00     0     0   numeric    591
## 17    vegetation_area    2648   35.25    0   0.00     0     0   numeric   1052
## 18       farming_area    5976   79.56    0   0.00     0     0   numeric    650
## 19 village_veget_area    1413   18.81    0   0.00     0     0   numeric   1377
## 20         total_area       8    0.11    0   0.00     0     0   numeric   1781
## 21         cause_type       0    0.00    0   0.00     0     0 character      4
\end{verbatim}

\begin{verbatim}
## # A tibble: 6 x 21
##      id region district municipality parish lat   lon   origin alert_date
##   <int> <chr>  <chr>    <chr>        <chr>  <chr> <chr> <chr>  <chr>     
## 1  7506 Trás-~ Bragança Mirandela    Carva~ 41:3~ 7:10~ firep~ 2015-07-18
## 2  7507 Beira~ Castelo~ Idanha-a-No~ Oledo  39:5~ 7:20~ firep~ 2015-08-07
## 3  7508 Entre~ Porto    Penafiel     São M~ 41:1~ 8:12~ firep~ 2015-08-08
## 4  7509 Entre~ Porto    Amarante     Telões 41:1~ 8:6:~ firep~ 2015-08-08
## 5  7510 Entre~ Braga    Celorico de~ Gémeos 41:2~ 8:0:~ firep~ 2015-08-08
## 6  7511 Ribat~ Santarém Ourém        Nossa~ 39:3~ 8:34~ firep~ 2015-08-08
## # ... with 12 more variables: alert_hour <chr>, extinction_date <chr>,
## #   extinction_hour <chr>, firstInterv_date <chr>, firstInterv_hour <chr>,
## #   alert_source <lgl>, village_area <dbl>, vegetation_area <dbl>,
## #   farming_area <dbl>, village_veget_area <dbl>, total_area <dbl>,
## #   cause_type <chr>
\end{verbatim}

Table @ref(tab:variables) describes all variables contained in the
\texttt{fires.raw} of the data set. Clearly, the type of some of
variables is incorrect and inconvenient for analysis and that was taken
care of in section @ref(data-cleaning).

\begin{longtable}[]{@{}llll@{}}
\caption{(\#tab:variables) List of variables present in the original
file from the fires2015train.csv data set.}\tabularnewline
\toprule
Variable & Type & Description &\tabularnewline
\midrule
\endfirsthead
\toprule
Variable & Type & Description &\tabularnewline
\midrule
\endhead
id & integer & id number &\tabularnewline
region & character & region name &\tabularnewline
district & character & district name &\tabularnewline
municipality & character & municipality name &\tabularnewline
parish & character & parish name &\tabularnewline
lat & character & latitude value &\tabularnewline
lon & character & longitude value &\tabularnewline
origin & character & how the fire started &\tabularnewline
alert\_date & character & date when fire started &\tabularnewline
alert\_hour & character & alert hour &\tabularnewline
extinction\_date & character & date of the end of fire &\tabularnewline
extinction hour & character & hour of the end of fire &\tabularnewline
firstInterv\_date & character & date of intervention &\tabularnewline
firstInterv\_hour & character & hour of intervention &\tabularnewline
alert\_source & logical & alert source &\tabularnewline
village\_area & numeric & village area affected &\tabularnewline
alert\_source & logical & alert source &\tabularnewline
village\_area & numeric & village area affected &\tabularnewline
vegetation\_area & numeric & vegetation area affected &\tabularnewline
farming\_area & numeric & farming area affected &\tabularnewline
village\_veget\_area & numeric & total village+veget affected
&\tabularnewline
total\_area & numeric & total area affected &\tabularnewline
cause\_type & character & cause of the fire &\tabularnewline
\bottomrule
\end{longtable}

\hypertarget{data-preparation}{%
\chapter{Data Preparation}\label{data-preparation}}

\hypertarget{data-cleaning}{%
\section{Data Cleaning:}\label{data-cleaning}}

We started to verify the dataset to identify and correct the mistakes or
errors in the data. First we tested if there were duplicated
observations.

\begin{verbatim}
## [1] 0
\end{verbatim}

The latitude variable had to be corrected because we observed that there
were incorrect values like a data mixed in it.

\begin{verbatim}
## # A tibble: 6 x 21
##      id region district municipality parish lat   lon   origin alert_date
##   <int> <chr>  <chr>    <chr>        <chr>  <chr> <chr> <chr>  <chr>     
## 1   102 Alent~ Évora    Reguengos d~ Campo  1900~ 07:3~ agric~ 2015-05-20
## 2   439 Alent~ Évora    Mora         Cabeç~ 1900~ 08:0~ agric~ 2015-05-16
## 3   486 Algar~ Faro     Silves       São B~ 1900~ 00:0~ firep~ 2015-05-21
## 4   963 Algar~ Faro     Lagos        Bensa~ 1900~ 00:0~ fire   2015-05-14
## 5  1512 Alent~ Évora    Vendas Novas Venda~ 1900~ 00:0~ fire   2015-05-19
## 6  1658 Algar~ Faro     Lagos        São S~ 1900~ 08:4~ agric~ 2015-07-26
## # ... with 12 more variables: alert_hour <chr>, extinction_date <chr>,
## #   extinction_hour <chr>, firstInterv_date <chr>, firstInterv_hour <chr>,
## #   alert_source <lgl>, village_area <dbl>, vegetation_area <dbl>,
## #   farming_area <dbl>, village_veget_area <dbl>, total_area <dbl>,
## #   cause_type <chr>
\end{verbatim}

The number of observations with wrong value in lat variable were
detected.

\begin{verbatim}
## [1] "There are 38 observations with wrong value '1900-01-01'"
\end{verbatim}

In the wrong values, an imputation of value was made based on another
observation that has the same region, district, municipality and parish.
After the imputation 8 NA values were assign to observations.

It was necessary to do a cleaning on lat and lon variables and convert
their contents from GPS coodinate to decimals.

\hypertarget{data-transforms}{%
\section{Data Transforms}\label{data-transforms}}

Changing the scale or distribution of variables.

\hypertarget{feature-engineering}{%
\section{Feature Engineering}\label{feature-engineering}}

Deriving new variables from available data.

\hypertarget{feature-selection}{%
\section{Feature Selection}\label{feature-selection}}

Identifying those input variables that are most relevant to the task.

\hypertarget{dimensionality-reduction}{%
\section{Dimensionality Reduction}\label{dimensionality-reduction}}

Creating compact projections of the data.

\hypertarget{exploratory-data-analysis}{%
\chapter{Exploratory Data Analysis}\label{exploratory-data-analysis}}

\hypertarget{conclusions-shortcomings-and-future-work}{%
\chapter{Conclusions, Shortcomings and Future
Work}\label{conclusions-shortcomings-and-future-work}}

\hypertarget{appendix}{%
\chapter{Appendix}\label{appendix}}

\end{document}
